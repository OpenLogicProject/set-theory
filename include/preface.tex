\chapter*{About}

This book is an Open Education Resource. It is written for students
with a little background in logic, and some high school mathematics.
It aims to scratch the tip of the surface of the philosophy of set
theory. By the end of this book, students reading it might have a
sense of:
\begin{enumerate}
	\item why set theory came about; 
	\item how to embed large swathes of mathematics within set theory + arithmetic;
	\item how to embed arithmetic itself within set theory;
	\item what the cumulative iterative conception of set amounts to;
	\item how one might try to justify the axioms of ZFC.
\end{enumerate}
The book grew out of a short course that I taught in the Cambridge Philosophy department. Before me, it was lectured by Luca Incurvati and Michael
Potter. In writing this book---and the course, more generally---I was
hugely indebted to both Luca and Michael. I hope this comes through in the text of the book itself; but I also want to offer both of them
my heartfelt thanks here.

Most of this book was originally released as \emph{Open
Set Theory}; this book is its successor. I have contributed all of the material for this book to the \href{http://openlogicproject.org}{\emph{Open Logic Project}}, where it is freely available. (\Crefrange{sfr:set::chap}{sfr:siz::chap} were drawn, with only tiny
changes, from previously existing material in the OLP; these changes are now also part of the OLP.) Please see
\href{http://openlogicproject.org/}{openlogicproject.org} for more
information.

\begin{flushright}
Tim Button
\\University College London
\\October 2021
\end{flushright}