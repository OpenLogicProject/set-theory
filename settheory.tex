% !TeX root = ./settheory-screen.tex
% settheory.tex
%
% driver file settheory.tex to produce text

\preto\OLEndChapterHook{\IfFileExists{include/summary-\thechapter}
        {\section*{Summary}\addcontentsline{toc}{section}{Summary}
        \let\emph\textbf\input{include/summary-\thechapter}\let\emph\textit}{}}

\problemsperchapter

\allowdisplaybreaks

\frontmatter

\OLPfrontmatter

\chapter*{About}

This book is an Open Education Resource. It is written for students
with a little background in logic, and some high school mathematics.
It aims to scratch the tip of the surface of the philosophy of set
theory. By the end of this book, students reading it might have a
sense of:
\begin{enumerate}
	\item why set theory came about; 
	\item how to embed large swathes of mathematics within set theory + arithmetic;
	\item how to embed arithmetic itself within set theory;
	\item what the cumulative iterative conception of set amounts to;
	\item how one might try to justify the axioms of ZFC.
\end{enumerate}
The book grew out of a short course that I taught in the Cambridge Philosophy department. Before me, it was lectured by Luca Incurvati and Michael
Potter. In writing this book---and the course, more generally---I was
hugely indebted to both Luca and Michael. I hope this comes through in the text of the book itself; but I also want to offer both of them
my heartfelt thanks here.

Most of this book was originally released as \emph{Open
Set Theory}; this book is its successor. I have contributed all of the material for this book to the \href{http://openlogicproject.org}{\emph{Open Logic Project}}, where it is freely available. (\Crefrange{sfr:set::chap}{sfr:siz::chap} were drawn, with only tiny
changes, from previously existing material in the OLP; these changes are now also part of the OLP.) Please see
\href{http://openlogicproject.org/}{openlogicproject.org} for more
information.

\begin{flushright}
Tim Button
\\University College London
\\October 2021
\end{flushright}

\mainmatter

\part{Prelude}

\olchapter{his}{set}{History and Mythology}

To understand the philosophical significance of set theory, it will
help to have some sense of why set theory arose at all. To understand
that, it will help to think a little bit about the history and
mythology of mathematics. So, before we get started on discussing set
theory at all, we will start with a very brief ``history''. But we put
this in scare-quotes, because it is very brief, extremely selective,
and somewhat contestable.

\olimport*[history/set-theory]{infinitesimals}
\olimport*[history/set-theory]{limits}
\olimport*[history/set-theory]{pathologies}
\olimport*[history/set-theory]{mythology}

\section{Roadmap}

Part of the moral of the previous section is that the history of
mathematics was largely written by the victors. They had axes to
grind; philosophical and mathematical axes. Serious study of the
history of mathematics is seriously difficult (and rewarding), and the
Owl of Minerva takes flight only at dusk.

For all that, it's incontestable that the ``pathological'' results
involved the development of fascinating new mathematical tools, and a
re-thinking of the standards of mathematical rigour. For example, they
required thinking of the continuum (the ``real line'') in a particular
way, and thinking of functions as point-by-point maps. And, in the
end, the full development of all of these tools required the rigorous
development of set theory. The rest of this book will explain some of
that development.

\Olref[sfr][][]{part} will present a version of \emph{na\"ive} set
theory, which is easily sufficient to develop all of the mathematics
just described. This will take a while. But, by end of
\olref[sfr][][]{part} we will be in a position to understand how to
treat real numbers as certain sets, and how to treat functions on
them---including space-filling curves---as \emph{further} sets. 

But the \emph{na\"ivety} of this set theory will emerge in
\olref[sth][][]{part}, as we encounter set-theoretic paradoxes, and
the felt need to describe things much more precisely. At this point,
we will need to develop an axiomatic treatment of sets, which we can
use to recapture all of our na\"ive results, whilst (hopefully)
avoiding paradoxes. (The Owl of mathematical rigour takes flight only
at dusk, too.)

\OLEndChapterHook

\olpart{sfr}{Na\"ive Set Theory}

\section*{Introduction to \Olref[sfr][][]{part}}

In \olref[sfr][][]{part}, we will consider sets in a na\"ive, informal
way. \Olref[sfr][set][]{chap} will introduce the basic idea, that sets
are collections considered extensionally, and will introduce some very
basic operations. Then \crefrange{sfr:rel::chap}{sfr:fun::chap} will
explain how set theory allows us to speak about relations and
(therefore) functions. 

\Crefrange{sfr:siz::chap}{sfr:infinite::chap} will then consider some of the
early achievements of na\"ive set theory. In
\olref[sfr][siz][]{chap}, we explore how to compare sets with regard
to their size. In \olref[sfr][arith][]{chap}, we explore how one
might reduce the integers, rationals, and reals to set theory plus
basic arithmetic. In \olref[sfr][infinite][]{chap}, we consider how one
might implement basic arithmetic within set theory. 

To repeat, all of this will be done \emph{na\"ively}. But everything
we do in \olref[sfr][][]{part} can be done perfectly rigorously, in
the formal set theory which we introduce in \olref[sth][][]{part}.

\olchapter{sfr}{set}{Getting Started}

\olimport*[sets-functions-relations/sets]{basics}
\olimport*[sets-functions-relations/sets]{subsets}
\olimport*[sets-functions-relations/sets]{important-sets}
\olimport*[sets-functions-relations/sets]{unions-and-intersections}
\olimport*[sets-functions-relations/sets]{pairs-and-products}
\olimport*[sets-functions-relations/sets]{russells-paradox}

\OLEndChapterHook

\olchapter{sfr}{rel}{Relations}

\olimport*[sets-functions-relations/relations]{relations-as-sets}
\olimport*[sets-functions-relations/relations]{reflections}
\olimport*[sets-functions-relations/relations]{special-properties}
\olimport*[sets-functions-relations/relations]{equivalence-relations}
\olimport*[sets-functions-relations/relations]{orders}
\olimport*[sets-functions-relations/relations]{operations}

\OLEndChapterHook

\olchapter{sfr}{fun}{Functions}

\olimport*[sets-functions-relations/functions]{function-basics}
\olimport*[sets-functions-relations/functions]{function-kinds}
\olimport*[sets-functions-relations/functions]{functions-relations}
\olimport*[sets-functions-relations/functions]{inverses}
\olimport*[sets-functions-relations/functions]{composition}

\OLEndChapterHook

\olchapter{sfr}{siz}{The Size of Sets}

\olimport*[sets-functions-relations/size-of-sets]{introduction}
\olimport*[sets-functions-relations/size-of-sets]{enumerability-alt}
\olimport*[sets-functions-relations/size-of-sets]{zig-zag}
\olimport*[sets-functions-relations/size-of-sets]{pairing}
\olimport*[sets-functions-relations/size-of-sets]{non-enumerability-alt}
\olimport*[sets-functions-relations/size-of-sets]{reduction-alt}
\olimport*[sets-functions-relations/size-of-sets]{equinumerous-sets}
\olimport*[sets-functions-relations/size-of-sets]{comparing-size}
\olimport*[sets-functions-relations/size-of-sets]{schroder-bernstein}
\olimport*[history/set-theory]{cantor-plane}
\olimport*[history/set-theory]{hilbert-curve}

\OLEndChapterHook

\olsetchapter{nosection}
\chapter{Arithmetization}

In \olref[his][set][]{chap}, we considered some of the historical
background, as to \emph{why} we even have set theory.
\Crefrange{sfr:set::chap}{sfr:siz::chap} then worked through through
some principles of na\"ive set theory. So we now understand,
na\"ively, how to construct relations and functions and compare the
sizes of sets, and \emph{things like that}. 

With this under our belts, we can approach some of the early
achievements of set theory, in reducing (in some sense) large chunks
of mathematics to set theory and arithmetic. That is the aim of this
chapter.

\olimport*[sets-functions-relations/arithmetization]{arithmetization}

\chapter{Infinite Sets}

In the previous chapter, we showed how to construct a bunch of
things---integers, rationals, and reals---assuming some na\"{i}ve set
theory and the natural numbers. The question for this chapter is: Can
we construct the set of natural numbers \emph{itself} using set
theory?

\olimport*[sets-functions-relations/infinite]{infinite}
\olresetchapter

\OLEndPartHook

\olpart{sth}{The Iterative Conception}

\section*{Introduction to \Olref[sth][][]{part}}

\Olref[sfr][][]{part} discussed sets in a na\"{i}ve, informal way. It
is now time to tighten this up, and provide a formal theory of sets.
That is the aim of \olref[sth][][]{part}. 

Our formal theory is a first-order theory with just one two-place
predicate, $\in$. We will lay down several axioms that govern the
behaviour of the membership relation. However, we will introduce these
axioms only as we need them, and consider how we might justify them as
we encounter them. As a result, we will introduce our axioms
\emph{throughout} the ensuing chapters.

It might, though, be helpful for the reader to have a list of all the
axioms in one place. So, here are \emph{all} the axioms that we will
consider in \olref[sth][][]{part}. As in \olref[sfr][][]{part}, the
choice of lowercase and uppercase letters is guided only by
readability:

\begin{defish}
\emph{Extensionality.} 
\[\forall A \forall B(\forall x(x \in A \liff x \in B) \lif A = B)\]
\end{defish}

\begin{defish}
\emph{Union.} 
\[
\forall A \exists U \forall x(x \in U \liff (\exists b \in A)x \in b),
\]
i.e., $\bigcup A$ exists for any set~$A$.
\end{defish}

\begin{defish}
\emph{Pairs.} 
\[
\forall a \forall b \exists P \forall x (x \in P \liff
(x = a \lor x = b)),
\]
i.e., $\{a, b\}$ exists for any $a$ and~$b$.
\end{defish}

\begin{defish}
\emph{Powersets.}
\[
\forall A \exists P \forall x(x \in P \liff (\forall z \in x)z \in A),
\]
i.e., $\Pow{A}$ exists for any set~$A$.
\end{defish}

\begin{defish}
\emph{Infinity.}
\begin{multline*}
\exists I((\exists o \in I)\forall x(x \notin o) \land {}\\
(\forall x \in I)(\exists s \in I)\forall z(z \in s \liff (z \in x \lor z = x))),
\end{multline*}
i.e., there is a set with $\emptyset$ as !!a{element} and which 
is closed under $x \mapsto x \cup \{x\}$.
\end{defish}

\begin{defish}
\emph{Foundation.}
\[
\forall A(\forall x\, x \notin A \lor (\exists b \in A)(\forall x \in A)x \notin b),
\]
i.e., $A = \emptyset$ or $(\exists b \in A)A\cap b = \emptyset$.
\end{defish}

\begin{defish}
\emph{Well-Ordering.} 
For every set $A$, there is a relation that well-orders $A$.
(Writing this one out in first-order logic is too painful to bother with.)
\end{defish}

\begin{defish}
\emph{Separation Scheme.}
For any formula $\phi(x)$ which does not contain~``$S$'':
\[
\forall A \exists S \forall x(x \in S \liff (\phi(x) \land x \in A)),
\]
i.e., $\Setabs{x \in A}{\phi(x)}$ exists for any set~$A$.
\end{defish}

\begin{defish}
\emph{Replacement Scheme.} For any formula $\phi(x, y)$ which does not contain~``$B$'':
\[
\forall A((\forall x \in A)\lexists![y][\phi(x,y)] \lif \exists B \forall y (y \in B \liff (\exists x \in A)\phi(x,y))),
\]
i.e., $\Setabs{y}{(\exists x \in A)\phi(x,y)}$ exists for any $A$, if $\phi$ is ``functional.''
\end{defish}

In both schemes, the formulas may contain parameters. Indeed,
throughout \olref[sth][][]{part}, we follow the convention that any
formula can contain parameters. (See
\olref[sfr][infinite][induction]{sec} for a reminder of what it means
to say that a formula may contain parameters.)

\begin{description}
\item[$\Zminus$] is Extensionality, Union, Pairs, Powersets, Infinity, Separation.
\item[$\Z$] is $\Zminus$ plus Foundation.
\item[$\ZFminus$] is $\Z$ plus Replacement.
\item[$\ZF$] is $\ZFminus$ plus Foundation.
\item[$\ZFC$] is $\ZF$ plus Well-Ordering.
\end{description}

\olimport*[set-theory/story]{story}

\chapter{Steps towards $\Z$}

In the previous chapter, we considered the iterative conception of
set. In this chapter, we will attempt to extract most of the axioms of
Zermelo's set theory, i.e.,~$\Z$. The approach is \emph{entirely}
inspired by \citet{Boolos1971},  \citet{Scott1974}, and
\citet{Shoenfield:AST}. 

\olsetchapter{nosection}
\olimport*[set-theory/z]{z}
\olresetchapter

\olimport*[set-theory/ordinals]{ordinals}

\olimport*[set-theory/spine]{spine}

\olimport*[set-theory/replacement]{replacement}

\olimport*[set-theory/ord-arithmetic]{ord-arithmetic}

\olimport*[set-theory/cardinals]{cardinals}

\chapter{Cardinal Arithmetic}

In \olref[sth][cardinals][]{chap}, we developed a theory of cardinals.
Our next step is to outline a theory of cardinal arithmetic. This
chapter briefly summarises some of the elementary facts, and then
points to some of the difficulties, which turn out to be fascinating
and philosophically rich.

\olsetchapter{nosection}
\olimport*[set-theory/card-arithmetic]{card-arithmetic}
\olresetchapter

\olimport*[set-theory/choice]{choice}

\stopproblems
\def\ifproblems#1{}

\appendix

\def\figurename{Fig.}
\setlength{\olphotowidth}{.45\textwidth}

\chapter{Biographies}

\olimport*[history/biographies]{georg-cantor}

\olimport*[history/biographies]{kurt-goedel}

\olimport*[history/biographies]{bertrand-russell}

\olimport*[history/biographies]{alfred-tarski}

\olimport*[history/biographies]{ernst-zermelo}

\backmatter

\clearpage

\photocredits

\bibliographystyle{\olpath/bib/natbib-oup}
\bibliography{\olpath/bib/open-logic}

\olimport*{\olpath/content/open-logic-about}